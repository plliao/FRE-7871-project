\documentclass[sigconf]{acmart}

\usepackage{url}

\begin{document}

\title{Reproduce Stock Price Prediction Tasks with News Analysis}
\subtitle{NYU FRE-7871 Final Project Proposal}

\author{Pei-Lun Liao}
\affiliation{%
  \institution{New York University}
}
\email{pll273@nyu.edu}

\begin{abstract}
In this project, selected related works will be reproduced with different financial news dataset. The goal is to understand techniques and challenges in stock price prediction.
\end{abstract}
\maketitle

\section{Introduction}
Nowadays, traders apply machine learning technique to extract information from financial news data. The information can be used to boost the accuracy of stock price
prediction \cite{AZFinText, Ding2014, Ding2015}. The project aims to reproduce selected related works \cite{AZFinText}. Hence, people can learn the challenge and technique
that are crucial in stock price prediction.

\section{Approaches}
\subsection{Data}
Data quality is crucial to make the accurate prediction \cite{stanford}. Unfortunately, it is not easy to find a publicly available financial news dataset. For example, the public financial
news dataset published in \cite{Ding2014} was currently unavailable due to license issue \cite{fn}. Another corpus available online requires membership and subscription fee \cite{nyt}.
Also, it is not practical to crawl dataset from business news media in this one month project. Most of the time will be spent on data collecting, cleaning, and processing instead of
understanding the technique that works for stock price prediction.

Fortunately, an available financial news dataset was found on Webhose.io \cite{data}. The data was crawled from the Internet from July to October in 2015. 47,851 news articles were collected
in machine-readable format. However, there is no paper shows the dataset could help stock price prediction.

\subsection{Plan}
\subsubsection{Stage 1: bag-of-words}
	The goal in stage 1 is reproducing the experiment result in AZFinText system \cite{AZFinText}. AZFinText represented article in the bag-of-words with only proper nouns. The
	dataset were Yahoo Finance news articles and the S \& P 500 index. Reproducing the experiment in AZFinText can prove that the data from Webhose.io also works for stock
	price prediction. Moreover, we examined whether only using proper nouns helps the performance.

	The experiment setup will follow the AZFinText paper. We will represent articles in the bag-of-words and use SVR to predict stock price. Finally, we evaluate the result by the rate of return.
	The strategy is buying the stock as the predicted price is greater than or equal to 1\% movement from the stock price at the time the article was released. Then, sell the stock after 20 minutes.
	In stage 1, we will have four different results to compare.
	\\
	\begin{itemize}
		\item 1. Stock price
		\item 2. Stock price + News in bag-of-words
		\item 3. Stock price + News in bag-of-words without stopwords
		\item 4. Stock price + News in bag-of-words with only proper nouns
	\end{itemize}
	The expected performance will be 4 > 3 > 2 > 1 as the paper described.

\subsubsection{Stage 2: word and sentence representation}
	In this stage, We are curious how modern deep learning techniques like word2vec, seq2seq, and CNN-LSTM language models could help improve the performance.  
	\begin{itemize}
		\item 5. Stock price + News feature in average word2vec \cite{word2vec1, word2vec2}
		\item 6. Stock price + News feature in Skip-Thought vector\cite{skip}
		\item 7. Stock price + News feature in CNN-LSTM encoding \cite{CNN2RNN} (may try to find pre-trained model)
	\end{itemize}
	The performance in stage 2 should be better than the one in stage 1.

\bibliographystyle{ACM-Reference-Format}
\bibliography{citation} 
\end{document}
